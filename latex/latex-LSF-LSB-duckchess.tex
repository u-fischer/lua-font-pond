% !Mode:: "TeX:UTF-8:Main"
% Date: 26.05.2018
% Description: use duckchess with chessboard
% open type feature
% Requires: - luafont-duckchess.lua, should be in a texmf tree in fonts/misc
%           - luafont-duckchess-LSB2.lua  ""
%           - duckchess-glyphs.pdf
%          
% open question:
% -how to call the font with fontspec commands.
%- can one avoid the terminal messages regarding the graphics inclusion
\documentclass[parskip=half-,paper=A5]{scrartcl}
\usepackage[LSB2]{fontenc}
\usepackage{xskak}
\usepackage{fontspec}

%The font has not sensible piece background so should at best be used on a colored board

\DeclareFontFamily{LSB2}{duckchess}{}
\DeclareFontShape{LSB2}{duckchess}{m}{n}{<-> 
 "[\directlua{tex.sprint(kpse.find_file("luafont-duckchess-LSB2.lua","misc\space fonts"))}]:mode=node;+liga;original=roboto" }{}

\DeclareFontFamily{LSF}{duckchess}{}
\DeclareFontShape{LSF}{duckchess}{m}{n}{<-> "
[\directlua{tex.sprint(kpse.find_file("luafont-duckchess.lua","misc\space fonts"))}]:mode=node;+liga;original=roboto" }{}


\usepackage[sfdefault,scale=0.8]{roboto}

\begin{document}
\setfigfontfamily{duckchess}
\setchessboard{boardfontfamily=duckchess,boardfontencoding=LSB2,
 color=white,
 trimtocolor=white,
 colorbackboard,
 color=gray!60,
 trimtocolor=black,
 colorbackboard}

\newchessgame
\mainline{%
1.d4 Nf6 2.c4 g6 3.Nc3 Bg7 4.e4 d6 5.Nge2 O-O 6.Ng3 c6 7.Be2 a6 8.a4
a5 9.h4 h5 10.Be3 Na6 11.f3 e5 12.d5 Nd7 13.Nf1 Ndc5 14.Nd2 Qb6
15.Qb1 Nb4 16.Nb3 Ncd3+ 17.Kd2 Qxe3+}

\raggedright
\xskakloop[step=6,showlast]{%
\begin{tabular}{c}
\chessboard[smallboard,
setfen=\xskakget{nextfen}]
\\
\xskakget{opennr}\xskakget{lan}%
\end{tabular}\quad}%

\end{document}

