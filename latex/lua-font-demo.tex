% !Mode:: "TeX:UTF-8:Main"
% Date: 13.05.2018
% Description: demonstrates the use of a font defined in a lua file
% the vowels are replaced by rules or lines.
% The basefont can be set with the key original
% Requires: latex-font-demo-rule.lua, which defines a font
% Source/Links:
% fonts-mkiv.pdf
% https://www.mail-archive.com/ntg-context@ntg.nl/msg87811.html
% Remarks:
% the original lua code tried to add an /Actualtext around
% the rule to get correct copy&paste. As this doesn't work with
% rules it was removed here.
%

\documentclass{article}
\begin{document}

% the value of "original" should be a name, not a filename!

% This syntax works without the database and so doesn't trigger a reload.
% but it doesn't work with files in the texmf-tree

\font\testluafont=[latex-font-demo-rule.lua]:mode=node;+liga;original=arial
{\testluafont So when we use it we get text ffi}

\font\testluafont=[latex-font-demo-rule.lua]:mode=node;+liga;option=line} at 30pt
{\testluafont So when we use it we get text ffi}


% This syntax works with files in the texmf tree (in the tex-folder) but triggers
% luaotfload | db : Reload initiated (formats: otf,ttf,ttc); reason: "File not found:
% latex-font-demo-rule.".
% This can be avoided by using the local luaotfload-database.lua (in texmf/tex/luatex)
% It would be more logical if the lua-files where in fonts (e.g. fonts/misc or fonts/opentype
% but this seems to need more changes in the luaotfload-files.

\font\testluafont={file:latex-font-demo-rule-texmf-tex.lua:mode=node;+liga;original=arial}
{\testluafont So when we use it we get text ffi}




\end{document}

