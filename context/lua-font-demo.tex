% !Mode:: "TeX:UTF-8:Main"
% Date: 13.05.2018
% Description: demonstrates the use of a font defined in a lua file
% Requires: context-font-demo-rule.lua, which defines a font
% Source/Links:
% fonts-mkiv.pdf
% open questions:
% the original lua uses pdf:direct which gives wrong nesting
% pdf:page is better but there are still BT/ET pairs too much
% probably from the context backends code
% copy & paste doesn't work as expected: /Actualtext doesn't work around simple rules,
% see e.g. https://tex.stackexchange.com/a/397609/2388

% use to uncompress the pdf:
\pdfvariable compresslevel =0

\definefontfeature
  [myrulefont]
  [default]
  [original=file:texgyrepagella-regular.otf,
   option=line]
\definefont
  [MyRuleFont]
  [file:context-font-demo-rule.lua*myrulefont]

\starttext
{\MyRuleFont So when we use it we get text }

%low-level definition
%at least one feature (e.g. mode=node) is needed

\font\testluafont={file:context-font-demo-rule.lua:mode=node;+liga;original=arial.ttf}
{\testluafont So when we use it we get text ffi}

\stoptext
