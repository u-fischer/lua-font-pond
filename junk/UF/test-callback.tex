\input luaotfload.sty

%% font resolver how-to:
%% simple example: short hand resolver that calls
%% the name resolver after translating the request
\directlua{

  %% first we define a dictionary of shorthands
  %% the CMU one contains subcategories
  local shorthands = {
    CMU = {
      default = "rm",
      rm = "CMU Serif",
      ss = "CMU Sans Serif",
    },
    AP  = "Antykwa Poltawskiego",
  }

  %% next we define some style shorthands
  local styles = {
    it = "italic",
    bf = "bold",
    bi = "bold italic",
  }

  %% then we need a function that resolves shorthands
  %% to hook into the callback
  local font_shorthands = function (spec)
    local req    = spec.name
    local parts  = string.explode(req, "!*")
    local family = parts[1]

    %% resolve the shorthand for the font name
    if not family then
      print("error: cannot resolve font", req)
      os.exit(1, true)
    else
      familydef = shorthands[family]
      if type(familydef) == "string" then
        family = familydef
      else
        family = familydef[parts[2]]
        if not family then
          family = familydef[familydef.default]
        end
      end
    end

    %% lookup the requested style
    local style = parts[3] or parts[2]
    if style and styles[style] then
      spec.style = styles[style]
    end

    spec.name = family

    %% rewrite the name field and call the standard
    %% name: resolver
    fonts.definers.resolvers.name(spec)
  end

  %% lastly, install the function in the callback
  luatexbase.add_to_callback(
    "luaotfload.resolve_font",
    font_shorthands,
    "user.font_shorthands")
}

\font\cmur  = "my:CMU!rm:+onum" at 1pc
\font\cmusi = "my:CMU!ss!it:+onum" at 1pc
\font\apr   = "my:AP"    at 12pt
\font\apbi  = "my:AP!bi" at 12pt

\cmur foo
{\cmusi bar} baz\par

{\apr whatever {\cmur back to normal}
 {\apbi emphatic!}
 this face again}\par

\bye